\documentclass[sigconf]{acmart}

\bibliographystyle{ACM-Reference-Format}

\setcopyright{none}
\copyrightyear{2020}
\acmYear{2020}

%%
%% PACKAGES HIER
%%

\usepackage[ngerman]{babel}
\usepackage{graphicx}
%%\usepackage{amssymb}
\usepackage{amsmath}

\usepackage{url}

\usepackage{geometry}
\geometry{letterpaper} 

\usepackage{listings}
\lstset{
  breaklines=true,
  literate=%
  {Ö}{{\"O}}1
  {Ä}{{\"A}}1
  {Ü}{{\"U}}1
  {ß}{{\ss}}1
  {ü}{{\"u}}1
  {ä}{{\"a}}1
  {ö}{{\"o}}1
}


\usepackage{subfiles} % letztes package!

%Styledefinition für Listings:
\definecolor{codegreen}{rgb}{0,0.6,0}
\definecolor{codegray}{rgb}{0.5,0.5,0.5}
\definecolor{codepurple}{rgb}{0.58,0,0.82}
\definecolor{backcolour}{rgb}{0.95,0.95,0.92}

\lstdefinestyle{mystyle}{
	backgroundcolor=\color{backcolour},   
	commentstyle=\color{codegreen},
	keywordstyle=\color{magenta},
	numberstyle=\tiny\color{codegray},
	stringstyle=\color{codepurple},
	basicstyle=\ttfamily\footnotesize,
	breakatwhitespace=false,         
	breaklines=true,                 
	captionpos=b,                    
	keepspaces=true,                 
	numbers=left,                    
	numbersep=5pt,                  
	showspaces=false,                
	showstringspaces=false,
	showtabs=false,                  
	tabsize=2
}

%Anwenden der Styledefiniton auf die Listings
\lstset{style=mystyle}

\acmConference[FWPM:Big Data Analysis]{Big Data Analysis}{Wintersemenster, 2019}{Hof}
\acmBooktitle{FWPM:Big Data Analysis, Wintersemenster, 2019, Hof}
\acmPrice{}
\acmISBN{}


\settopmatter{printacmref=false}
\settopmatter{printfolios=true}

\renewcommand\footnotetextcopyrightpermission[1]{}




\begin{document}

%%
%% TITLE TODO
%%

\title[Crypto Trading Recommender mittels Spark]{Crypto Trading Recommender mittels Spark}

\author{Tina Amann}
\email{tina.amann@hof-university.de}
\affiliation{
\institution{Hochschule Hof}
\streetaddress{Alfons-Goppel-Platz 1}
\city{Hof an der Saale}
\country{Germany}
}



\author{Daniel Eckardt}
\email{daniel.eckardt@hof-university.de}
\affiliation{
\institution{Hochschule Hof}
\streetaddress{Alfons-Goppel-Platz 1}
\city{Hof an der Saale}
\country{Germany}
}

\author{Jan Bernd Gaida}
\email{jan.gaida@hof-university.de}
\affiliation{
\institution{Hochschule Hof}
\streetaddress{Alfons-Goppel-Platz 1}
\city{Hof an der Saale}
\country{Germany}
}

\author{Patrick David Huget}
\email{patrick.huget@hof-university.de}
\affiliation{
\institution{Hochschule Hof}
\streetaddress{Alfons-Goppel-Platz 1}
\city{Hof an der Saale}
\country{Germany}
}

\author{Hannes Müller}
\email{hannes.mueller@hof-university.de}
\affiliation{
\institution{Hochschule Hof}
\streetaddress{Alfons-Goppel-Platz 1}
\city{Hof an der Saale}
\country{Germany}
}

\author{Alexander Puchta}
\email{alexander.puchta@hof-university.de}
\affiliation{
\institution{Hochschule Hof}
\streetaddress{Alfons-Goppel-Platz 1}
\city{Hof an der Saale}
\country{Germany}
}

\author{Leonardo Perak}
\email{leonardo.perak@hof-university.de}
\affiliation{
\institution{Hochschule Hof}
\streetaddress{Alfons-Goppel-Platz 1}
\city{Hof an der Saale}
\country{Germany}
}

\author{Sebastian Leuoth (Hrsg.)}
\email{sebastian.leuoth@hof-university.de}
\affiliation{
\institution{Hochschule Hof}
\streetaddress{Alfons-Goppel-Platz 1}
\city{Hof an der Saale}
\country{Germany}
}

%%
%% ABSTRACT TODO
%%

\begin{abstract}
Im Rahmen dieses hier vorgestellten Projekts galt es ein Trading Bot für Kryptowährungen mit Hilfe des Cluster Computing Frameworks Spark zu entwickeln. Hierfür wurden in einem aufgeteilten Entwicklungsprozess über mehrere Arbeitsschritte hinweg, die einzelnen Komponenten fertiggestellt. Bei Realisierung des Projektziels kamen Konzepte des Data Ware Housings zur Anwendung. Schlussendlich wurde vom Entwickler-Team ein Trading Bot entwickelt und implementiert, der durch die Kombination mehrerer verschiedener Indikatoren und einem damit unmittelbar zusammenhängenden Wahlverfahren entsprechend abgeleitete Handelsentscheidung auf Basis historischer, aufbereiteter Daten trifft. 
\end{abstract}


%%
%% KEYWORDS TODO
%%

%%\keywords{datasets, neural networks, gaze detection, text tagging}



\maketitle

%%\pagestyle{plain}
\renewcommand{\shortauthors}{}

%%
%% EINLEITUNG (Hab ich so von https://github.com/sleuoth-hof/trading_2019/blob/master/Abgabe/acmart-master/samples/sample-sigconf.tex hierher kopiert)
%%
%%\subfile{Einleitung/abstract.tex}
\subfile{Einleitung/einleitung.tex}

\subfile{Einleitung/relWork.tex}


%%
%% TODO REIHENFOLGE / STRUKTUR NICHT FINAL
%%

\subfile{Datenaufbereitung/api.tex}
\subfile{Datenaufbereitung/interpolation.tex}
%%
%% SECTION PPO - Percentage Price Oscillator
%%
\subfile{ppo/ppo.tex}

%%
%% SECTION NEO
%%
\subfile{NEO_Indikator/NEO_Indikator.tex}

%%
%% SECTION KURSVERGLEICH
%%
\subfile{kursvergleich/kursvergleich.tex}

%%
%% SECTION  Wahlverfahren und Transaktionsempfehlung
\subfile{wahlverfahren/Doku.tex}
%% SECTION BOT - Trading Bot
%%
\subfile{bot/bot.tex}

%%
%% LITERATURVERZEICHNIS
%%
\bibliography{master-doc}

%%
%% ABBILDUNGSVERZEICHNIS
%%
\listoffigures

%%
%% TABELLENVERZEICHNIS
%%
\listoftables

\end{document}
