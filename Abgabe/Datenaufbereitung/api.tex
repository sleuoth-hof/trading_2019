\section{Datenbezug über die API}

Gemäß des in der Einleitung beschriebenen Data Warehouse Gedanken müssen zunächst geeignete Datenquellen herangezogenen werden, bevor mit den daraus bezogenen Informationen weiter verfahren werden kann. Im Fall dieses Projekts müssen daher geeignete Daten eingebunden werden, auf dessen Basis der entwickelte Bot arbeitet. Hierfür wurden die historischen Daten für eine größere Menge an Kryptowährungen gesammelt und dann aufbereitet. Die Projektgruppe arbeitete schlussendlich mit 416 verschiedenen Kryptowährungen. Die historischen Daten hierfür wurden mittels der von der Börse für Kryptowährungen bereitgestellten Schnittstelle bezogen, vgl. \cite{bitfinex}. Um von dieser effizient für die eigenen Zwecke Gebrauch zu machen, wurde sich an entsprechenden Dokumentationen und allgemein zu findenden Tutorials orientiert, vgl. \cite{kaggle}.

Hierbei wurden für jedes Währungspapier die historischen Daten innerhalb einer bestimmten Zeitperiode geladen und gespeichert. Es wurde ein beliebiges Startdatum gewählt. Als Enddatum wurde der Tag vor demjenigen Tag gewählt an welchem die Funktion zuletzt ausgeführt wurde. Für diesen Zeitraum wurde täglich der jeweils letzte Wert („End of day“) abgefragt. Nachdem die Daten für die entsprechenden Daten gesammelt wurden, wurden sie zur Interpolation weitergereicht.